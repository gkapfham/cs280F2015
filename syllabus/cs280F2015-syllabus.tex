%!TEX root=cs280F2015_syllabus.tex

% Typical usage (all UPPERCASE items are optional):
%       \input 580pre
%       \begin{document}
%       \MYTITLE{Title of document, e.g., Lab 1\\Due ...}
%       \MYHEADERS{short title}{other running head, e.g., due date}
%       \PURPOSE{Description of purpose}
%       \SUMMARY{Very short overview of assignment}
%       \DETAILS{Detailed description}
%         \SUBHEAD{if needed} ...
%         \SUBHEAD{if needed} ...
%          ...
%       \HANDIN{What to hand in and how}
%       \begin{checklist}
%       \item ...
%       \end{checklist}
% There is no need to include a "\documentstyle."
% However, there should be an "\end{document}."
%
%===========================================================
\documentclass[11pt,twoside,titlepage]{article}
%%NEED TO ADD epsf!!
\usepackage{threeparttop}
\usepackage{graphicx}
\usepackage{latexsym}
\usepackage{color}
\usepackage{listings}
\usepackage{fancyvrb}
%\usepackage{pgf,pgfarrows,pgfnodes,pgfautomata,pgfheaps,pgfshade}
\usepackage{tikz}
\usepackage[normalem]{ulem}
\tikzset{
    %Define standard arrow tip
%    >=stealth',
    %Define style for boxes
    oval/.style={
           rectangle,
           rounded corners,
           draw=black, very thick,
           text width=6.5em,
           minimum height=2em,
           text centered},
    % Define arrow style
    arr/.style={
           ->,
           thick,
           shorten <=2pt,
           shorten >=2pt,}
}
\usepackage[noend]{algorithmic}
\usepackage[noend]{algorithm}
\newcommand{\bfor}{{\bf for\ }}
\newcommand{\bthen}{{\bf then\ }}
\newcommand{\bwhile}{{\bf while\ }}
\newcommand{\btrue}{{\bf true\ }}
\newcommand{\bfalse}{{\bf false\ }}
\newcommand{\bto}{{\bf to\ }}
\newcommand{\bdo}{{\bf do\ }}
\newcommand{\bif}{{\bf if\ }}
\newcommand{\belse}{{\bf else\ }}
\newcommand{\band}{{\bf and\ }}
\newcommand{\breturn}{{\bf return\ }}
\newcommand{\mod}{{\rm mod}}
\renewcommand{\algorithmiccomment}[1]{$\rhd$ #1}
\newenvironment{checklist}{\par\noindent\hspace{-.25in}{\bf Checklist:}\renewcommand{\labelitemi}{$\Box$}%
\begin{itemize}}{\end{itemize}}
\pagestyle{threepartheadings}
\usepackage{url}
\usepackage{wrapfig}
% removing the standard hyperref to avoid the horrible boxes
%\usepackage{hyperref}
\usepackage[hidelinks]{hyperref}
% added in the dtklogos for the bibtex formatting
\usepackage{dtklogos}
%=========================
% One-inch margins everywhere
%=========================
\setlength{\topmargin}{-.2in}
\setlength{\textheight}{8.75in}
\setlength{\oddsidemargin}{0in}
\setlength{\evensidemargin}{0in}
\setlength{\textwidth}{6.5in}
%===============================
%===============================
% Macro for document title:
%===============================
\newcommand{\MYTITLE}[1]%
   {\begin{center}
     \begin{center}
     \bf
     CMPSC 280\\Principles of Software Development\\
     Fall 2015
     \medskip
     \end{center}
     \bf
     #1
     \end{center}
}
%================================
% Macro for headings:
%================================
\newcommand{\MYHEADERS}[2]%
   {\lhead{#1}
    \rhead{#2}
    %\immediate\write16{}
    %\immediate\write16{DATE OF HANDOUT?}
    %\read16 to \dateofhandout
    \def \dateofhandout {August 25, 2015}
    \lfoot{\sc Handed out on \dateofhandout}
    %\immediate\write16{}
    %\immediate\write16{HANDOUT NUMBER?}
    %\read16 to\handoutnum
    \def \handoutnum {1}
    \rfoot{Handout \handoutnum}
   }

%================================
% Macro for bold italic:
%================================
\newcommand{\bit}[1]{{\textit{\textbf{#1}}}}

%=========================
% Non-zero paragraph skips.
%=========================
\setlength{\parskip}{1ex}

%=========================
% Create various environments:
%=========================
\newcommand{\PURPOSE}{\par\noindent\hspace{-.25in}{\bf Purpose:\ }}
\newcommand{\SUMMARY}{\par\noindent\hspace{-.25in}{\bf Summary:\ }}
\newcommand{\DETAILS}{\par\noindent\hspace{-.25in}{\bf Details:\ }}
\newcommand{\HANDIN}{\par\noindent\hspace{-.25in}{\bf Hand in:\ }}
\newcommand{\SUBHEAD}[1]{\bigskip\par\noindent\hspace{-.1in}{\sc #1}\\}
%\newenvironment{CHECKLIST}{\begin{itemize}}{\end{itemize}}


\usepackage[compact]{titlesec}

\begin{document}
\MYTITLE{Syllabus}
\MYHEADERS{Syllabus}{}

\subsection*{Course Instructor}
Dr.\ Gregory M.\ Kapfhammer\\
\noindent Office Location: Alden Hall 108 \\
\noindent Office Phone: +1 814-332-2880 \\
\noindent Email: \url{gkapfham@allegheny.edu} \\
\noindent Twitter: \url{@GregKapfhammer} \\
\noindent Web Site: \url{http://www.cs.allegheny.edu/sites/gkapfham/}

\subsection*{Instructor's Office Hours}

\begin{itemize}
    \itemsep 0em
    \item Monday: 1:00 pm -- 2:30 pm (30 minute time slots)
    \item Tuesday: 2:30 pm -- 4:00 pm (15 minute time slots)
    \item Wednesday: 4:30 pm -- 5:00 pm (15 minute time slots)
    \item Thursday: 9:00 am -- 10:00 am (15 minute time slots) {\em and} \\ \hspace*{.69in} 2:30 pm -- 4:00 pm (15 minute time slots)
    \item Friday: 1:00 pm -- 2:30 pm (10 minute time slots) {\em and} \\ \hspace*{.49in} 4:30 pm -- 5:00 pm (5 minute time slots)
\end{itemize}

\noindent
To schedule a meeting with me during my office hours, please visit my Web site and click the ``Schedule'' link in the
top right-hand corner. Now, you can browse my office hours or schedule an appointment by clicking the correct link and
then reserving an open time slot.

\subsection*{Course Meeting Schedule}

Lecture, Discussion, Presentations, and Group Work: Tuesday and Thursday, 11:00 am -- 12:15 pm \\
Laboratory Session: Wednesday, 2:30 pm -- 4:20 pm \\
Final Examination: Friday, December 13, 2013 at 9:00 am

\subsection*{Course Catalogue Description}

\begin{quote}

    A study of the principles and concepts used in the specification, design, implementation, testing, and maintenance
    of large software systems. Topics include requirements elicitation and analysis, formal specification, software
    architectures, object-oriented design, software measurement, software testing and analysis, and evolution of a
    program.  Students practice the principles of software development by participating as group members in the creation
    of a significant software application. One laboratory per week. Prerequisites: Computer Science 210 and 220 or
    permission of the instructor. \\ {\em Offered in alternate years}.

\end{quote}

\subsection*{Course Objectives}

The process of developing software involves the application of a number of interesting theories, tools, techniques, and
methodologies.  In this class we will explore the phases of the major software development life cycles and examine the
tools, concepts, challenges, and open questions associated with each phase.  Throughout the semester, we will examine
the interplay between the theory and practice of software development.  We will delve into the details of software
specification, design, implementation, testing, and maintenance through a discussion of book chapters and articles from
the software engineering and software testing literature.  Moreover, students will enhance their ability to write about
software in a clear and concise fashion.  Finally, students will gain practical software development experience in
laboratory sessions and a final software development project.

\subsection*{Performance Objectives}

At the completion of this class, a student should be aware of the fundamental challenges associated with software
development.  Furthermore, students should be comfortable with a wide array of concepts, methodologies, techniques, and
tools that they can apply to the problem of developing large software systems.  However, a successful student will
emerge with more than an understanding of the tools (e.g., text editors, compilers, debuggers, integrated development
environments, and version control systems) that a software engineer uses.  A student also should have a fundamental
understanding of the major software life cycles and the activities that take place in each phase of these life cycles.
Finally, a student should have a basic understanding of some of the current research and the open research questions in
the field of software engineering.  After completing this class, a student should be equipped for further graduate study
in the fields of computer science and software engineering.  The student should also be able to participate in
real-world software development projects by adeptly using modern software tools and working with a team of developers.

\subsection*{Required Textbooks}

% Shari Lawrence Pfleeger and Joanne M. Atlee
%   Software Engineering: Theory and Practice (Fourth Edition)
%   ISBN-10: 0136061699
%   ISBN-13: 978-0136061694
%   Prentice Hall
%   Status: Required
%   25 copies

\noindent{\em Software Engineering: Theory and Practice}. Shari Lawrence Pfleeger and Joanne M. Atlee.
Fourth Edition, ISBN-10: 0136061699, ISBN-13: 978-0136061694, 792 pages, 2009. \\
(References to the textbook are abbreviated as ``SETP'' in the syllabus and on the Web site).

 % The Mythical Man-Month: Essays on Software Engineering, Anniversary
 %  Edition (2nd Edition) [Paperback]
 %  Author: Frederick P. Brooks
 %  Publisher: Addison-Wesley Professional; Anniversary edition (August 12, 1995)
 %  ISBN-10: 0201835959
 %  ISBN-13: 978-0201835953
 %  Status: Required
 %  25 copies

\noindent{\em The Mythical Man Month}. Frederick P.\ Brooks, Jr.
Second Edition, ISBN-10: 0201835959, ISBN-13: 978-0201835953, 336 pages, 1985. \\
(References to the textbook are abbreviated as ``MMM'' in the syllabus and on the Web site).

\noindent
Students who want to improve their technical writing skills may consult the following books.

\noindent{\em BUGS in Writing: A Guide to Debugging Your Prose}. Lyn Dupr\'e. Second Edition,  ISBN-10: 020137921X,
ISBN-13: 978-0201379211, 704 pages, 1998.

\noindent{\em Writing for Computer Science}.  Justin Zobel. Second Edition,  ISBN-10: 1852338024, ISBN-13:
978-1852338022, 270 pages, 2004.

\noindent
Along with reading the required books, you will be asked to study many additional articles from a wide variety of
conference proceedings, journals, and the popular press.

\subsection*{Class Policies}

\subsubsection*{Grading}

The grade that a student receives in this class will be based on the following categories. All percentages are
approximate and, if the need to do so presents itself, it is possible for the assigned percentages to change during the
academic semester.

\begin{center}
\begin{tabular}{ll}
Class Participation and Instructor Meetings & 5\% \\
First Examination & 15\% \\
Second Examination & 15\% \\
Final Examination & 20\% \\
Laboratory and Homework Assignments & 30\% \\
Final Project & 15\%
\end{tabular}
\end{center}

\vspace*{-.1in}
\noindent
These grading categories have the following definitions:
\vspace*{-.1in}


\begin{itemize}

    \item {\em Class Participation and Instructor Meetings}: All students are required to actively participate during
        all of the class sessions. Your participation will take forms such as answering questions about the required
        reading assignments, asking constructive questions of your group members, giving presentations, and leading a
        discussion session. Furthermore, all students are required to meet with the course instructor during office
        hours for a total of sixty minutes during the Fall 2013 semester.  These meetings must be scheduled through the
        course instructor's reservation system and documented on a meeting record that you submit on the day of the final
        examination. A student will receive an interim and final grade for this category.

    \item {\em First and Second Examinations}: The first and second interim examinations will cover all of the material
        in their associated module(s).  While the second examination is not cumulative, it will assume that a student has a
        basic understanding of the material that was the focus of the first examination.  The date for the first and
        second examinations will be announced at least one week in advance of the scheduled date.  Unless prior
        arrangements are made with the course instructor, all students will be expected to take these examinations on the
        scheduled date and complete the tests in the stated period of time.

    \item {\em Final Examination}: The final examination is a three-hour cumulative test.  By enrolling in this course,
        students agree that, unless there are extenuating circumstances, they will take the final examination at the
        time stated on the first page of the syllabus.

    \item {\em Laboratory and Homework Assignments}: These assignments invite students to explore the concepts, tools,
        and techniques that are associated with different phases of the software development life cycle.  All of the
        laboratory assignments require the use of the provided tools to design, implement, test, and maintain programs
        that solve important problems.  To ensure that students are ready to develop software in both other classes at
        Allegheny College and after graduation, the instructor will assign individuals to teams for each of the
        laboratory assignments.  Unless specified otherwise, each laboratory assignment will be due at the beginning of
        the next laboratory session.  Many of the laboratory assignments in this course will expect students to
        give both a presentation and a demonstration of the software that they specified, designed, implemented, tested,
        and documented.

        %%% Homework assignments will normally ask students to prepare short written documents reflecting on
        % facets of the software development life cycles.

    \item {\em Final Project}: This project will present you with the description of a problem and ask you to
        implement a full-featured solution using one or more programming languages and a wide variety of software
        development tools.  The final project in this class will require you to apply all of the knowledge and skills
        that you have accumulated during the course of the semester to solve a problem and, whenever possible,
        make your solution publicly available as a free and open-source tool.  The project will invite you to draw upon
        both your problem solving skills and your knowledge of programming languages and software engineering tools. The
        final project will be completed in groups assigned by the course instructor.

\end{itemize}

\subsubsection*{Assignment Submission}

All assignments will have a stated due date. The printed version of the assignment is to be turned in at the beginning
of the class on that due date; the printed materials must be dated and signed with the Honor Code pledge of all the
students in the group.  When the printed version is submitted, the electronic version of the assignment also must be
made available to the course instructor in a version control repository. Late assignments will be accepted for up to one
week past the assigned due date with a 10\% penalty. All late assignments must be submitted at the beginning of the
session that is scheduled one week after the due date. Unless special arrangements are made with the course instructor,
no assignments will be accepted after the late deadline. In addition to submitting the required deliverables for any
assignment completed in a group, students must turn in a one-page document that describes each group member's
contribution to the submitted deliverables.

\subsubsection*{Attendance}

It is mandatory for all students to attend the class and laboratory sessions. If you will not be able to attend a
session, then please see the course instructor at least one week in advance to describe your situation.  Students who
miss more than five unexcused classes, laboratory sessions, or group project meetings will have their final grade in the
course reduced by one letter grade. Students who miss more than ten of the aforementioned events will automatically fail
the course.

% \subsection*{Laboratory Attendance Policy}
%
% In order to acquired the proper skills in technical writing, critical reading, and the presentation of technical
% material, it is essential for students to have hands-on experience in a laboratory. Therefore, it is mandatory for all
% students to attend the laboratory sessions. If you will not be able to attend a laboratory, then please see the course
% instructors at least one week in advance in order to explain your situation. Students who miss more than two unexcused
% laboratories will have their final grade in the course reduced by one letter grade.  Students who miss more than four
% unexcused laboratories will automatically fail the course.
%

\subsubsection*{Use of Laboratory Facilities}

Throughout the semester, we will experiment with many different tools that software engineers use during the phases of
the software development life cycle.  The course instructor and the department's systems administrator have invested a
considerable amount of time to ensure that our laboratories support the completion of both the laboratory assignments and the
final project.  To this end, students are required to complete all assignments and the final project while using the
department's laboratory facilities. The course instructor and the systems administrator will only be able to devote a
limited amount of time to the configuration of a student's personal computer.

\subsubsection*{Class Preparation}

% The study of the computer science discipline is very challenging.  Students in this class will be challenged to learn
% the principles and practice of software development.  During the coming semester even the most diligent student will
% experience times of frustration when they are attempting to understand a challenging concept or complete a difficult
% laboratory assignment.  In many situations some of the material that we examine will initially be confusing : do not
% despair!  Press on and persevere!
%

In order to minimize confusion and maximize learning, students must invest time to prepare for class discussions and
lectures.  During the class periods, the course instructor will often pose demanding questions that could require group
discussion, the creation of a program or test suite, a vote on a thought-provoking issue, or a group presentation.
Only students who have prepared for class by reading the assigned material and reviewing the current assignments will be
able to effectively participate in these discussions.  More importantly, only prepared students will be able to acquire
the knowledge and skills that are needed to be successful in both this course and the field of software development.  In
order to help students remain organized and effectively prepare for classes, the course instructor will maintain a class
schedule with reading assignments and presentation slides.   During the class sessions students will also be required to
download, use, and modify programs, diagrams, and data sets that are made available through the course Web site.
Students who are not comfortable with compiling, editing, and running Java programs should see the course instructor.

\subsubsection*{Email}

Using your Allegheny College email address, I will sometimes send out class announcements about matters such as
assignment clarifications or changes in the schedule. It is your responsibility to check your email at least once a day
and to ensure that you can reliably send and receive emails. This class policy is based on the following statement in
{\em The Compass}, the college's student handbook.

\vspace*{-.1in}
\begin{quote}
``The use of email is a primary method of communication on campus. \ldots
All students are provided with a campus email account and address while
enrolled at Allegheny and are expected to check the account on a regular
basis.''
\end{quote}
\vspace*{-.15in}

\subsubsection*{Disability Services}

The Americans with Disabilities Act (ADA) is a federal anti-discrimination statute that provides comprehensive civil
rights protection for persons with disabilities.  Among other things, this legislation requires all students with
disabilities be guaranteed a learning environment that provides for reasonable accommodation of their disabilities.
Students with disabilities who believe they may need accommodations in this class are encouraged to contact Disability
Services at 332-2898.  Disability Services is part of the Learning Commons and is located in Pelletier Library.
Please do this as soon as possible to ensure that approved accommodations are implemented in a timely fashion.

\subsubsection*{Honor Code}

The Academic Honor Program that governs the entire academic program at Allegheny College is described in the Allegheny
Course Catalogue.  The Honor Program applies to all work that is submitted for academic credit or to meet non-credit
requirements for graduation at Allegheny College.  This includes all work assigned for this class (e.g., examinations,
laboratory assignments, and the final project).  All students who have enrolled in the College will work under the Honor
Program.  Each student who has matriculated at the College has acknowledged the following pledge:

\vspace*{-.1in}
\begin{quote}
I hereby recognize and pledge to fulfill my responsibilities, as defined in the Honor Code, and to maintain the
integrity of both myself and the College community as a whole.
\end{quote}
\vspace*{-.15in}

\subsection*{Welcome to a Software Engineering Adventure}

In reference to software, Frederick Brooks, Jr.\ wrote in Chapter One of MMM, ``The magic of myth and legend has come true
in our time.'' Software is a pervasive aspect of our society that changes how we think and act.  High quality software
also has the potential to positively influence the lives of people. Moreover, the specification, design, implementation,
testing, maintenance, and documentation of software are exciting and rewarding activities!  At the start of this class,
I invite you to pursue this adventure in software engineering with enthusiasm and vigor.

\end{document}
