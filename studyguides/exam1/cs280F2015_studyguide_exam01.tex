\input{111pre}
\begin{document}
\MYTITLE{Examination 1 Study Guide \\ Delivered: Tuesday, September 22, 2015 \\ Examination: Thursday, October 1, 2015, 11:00 am}

\section*{Introduction}

This course will have its first examination on Friday, September 25, 2015 from 11:00 to 12:15 pm. The examination will
be ``closed notes'' and ``closed book'' and it will cover the following content. Please review the ``Course Schedule''
on the Web site for the course to see the content and slides that we have covered in the first module. You may post
questions about this material to Slack.

\begin{itemize}

  \itemsep 0in

  \item Chapters One through Three in SETP (i.e., introduction to the software engineering lifecycle)

  \item Chapters One through Three in MMM (i.e., challenges and solutions in software engineering)

  \item Knowledge of the basic commands necessary for using {\tt git} and Bitbucket; basic understanding of the Markdown
    syntax and the use of associated command-line tools such as {\tt pandoc}

  \item Your class notes, class activities, lecture slides, and the first four laboratory assignments

\end{itemize}

\vspace*{-.05in}
\noindent The examination will include a mix of questions that may require you to draw and/or comment on a diagram,
write a short answer, explain and/or write a source code segment, provide and comment on a list of points. The emphasis
will be on the following list of illustrative topics:

\vspace*{-.05in}
\begin{itemize}

  \itemsep 0in

  \item The state-of-the-art and the key challenges within the field of software engineering, with a focus on the steps
    of problem solving and the meaning of terms like ``defect'' and ``quality''.

  \item The phases of the software development lifecycle and the ways in which different software process models (e.g.,
    the spiral model or the V model) connect and interpret these phases.

  \item The key strengths and weaknesses of the different software development process models (e.g., one drawback of
    the waterfall model is its focus on documents and its lack of explicit iteration).

  \item Key terms such as ``verification'' and ``validation'' and ``incremental'' and ``iterative''.

  \item How to use activity graphs to track progress and plan a software development project. Additionally, an
    understanding of the ways in which managers will estimate the deadlines for completing a software system (e.g.,
    using data mining algorithms to predict characteristics such as project costs and the likelihood of an on-time
    completion).

\end{itemize}

\vspace*{-.05in}
\noindent Minimal partial credit may be awarded for the questions that require a student to write a short answer. You
are strongly encouraged to write short, precise, and correct responses to all of the questions. When you are taking the
examination, you should do so as a ``point maximizer'' who first responds to the questions that you are most likely to
answer correctly for full points. Please keep the time limitation in mind as you are absolutely required to submit the
examination at the end of the class period unless you have written permission for extra time from a member of the
Learning Commons. Students who do not submit their examination on time will have their overall point total reduced.
Please see the course instructor if you have questions about any of these policies.

\section*{Reminder Concerning the Honor Code}

\noindent Students are required to fully adhere to the Honor Code during the completion of this examination. More details about
the Allegheny College Honor Code are provided on the syllabus. Students are strongly encouraged to carefully review the
full statement of the Honor Code before taking this examination.

\noindent The following provides you with a review of Honor Code statement from the course syllabus:

The Academic Honor Program that governs the entire academic program at Allegheny College is described in the Allegheny
Academic Bulletin.  The Honor Program applies to all work that is submitted for academic credit or to meet non-credit
requirements for graduation at Allegheny College.  This includes all work assigned for this class (e.g., examinations,
laboratory assignments, and the final project).  All students who have enrolled in the College will work under the Honor
Program.  Each student who has matriculated at the College has acknowledged the following pledge:

\vspace*{-.11in}
\begin{quote}
  I hereby recognize and pledge to fulfill my responsibilities, as defined in the Honor Code, and to maintain the
  integrity of both myself and the College community as a whole.
\end{quote}
\vspace*{-.11in}

% \noindent It is understood that an important part of the learning process in any course, and particularly one in
% computer science, derives from thoughtful discussions with teachers and fellow students.  Such dialogue is encouraged.
% However, it is necessary to distinguish carefully between the student who discusses the principles underlying a problem
% with others and the student who produces assignments that are identical to, or merely variations on, someone else's
% work.  While it is acceptable for students in this class to discuss their programs, data sets, and reports with their
% classmates, deliverables that are nearly identical to the work of others will be taken as evidence of violating the
% \mbox{Honor Code}.

\vspace*{-.15in}
\section*{Detailed Review of Content}
\vspace*{-.1in}

The listing of topics in the following subsections is not exhaustive; rather, it serves to illustrate the types of
concepts that students should study as they prepare for the examination. Please see the course instructor during office
hours if you have questions about any of the content listed in this section.

\vspace*{-.1in}
\subsection*{Chapter One}

\begin{itemize}

  \itemsep 0in
  \item Basic understanding of computer hardware and software
  \item Computer number systems (e.g., binary and decimal)
  \item Purpose for and steps of the fetch-decode-execute cycle in the CPU
  \item Layout of and access techniques for computer memory
  \item Knowledge of computer networking methods and programs
  \item Basic syntax and semantics of the Java programming language
  \item Input(s) and output(s) of the Java compiler and virtual machine

\end{itemize}

\vspace*{-.2in}
\subsection*{Chapter Two}

\begin{itemize}

  \itemsep 0in
  \item Using escape sequences in the output of Java programs
  \item Ways to perform input and output in a Java program
  \item The variety of data types available to Java programmers
  \item The declaration of and assignment of values to variables
  \item Operators and operator precedence in Java expressions
  \item Techniques for converting variables from one data type to another
  \item Computer graphics and related topics such as pixels and screen resolution
  \item The use of the RGB system for specifying colors in Java programs

\end{itemize}


\end{document}
