% Typical usage (all UPPERCASE items are optional):
%       \input 580pre
%       \begin{document}
%       \MYTITLE{Title of document, e.g., Lab 1\\Due ...}
%       \MYHEADERS{short title}{other running head, e.g., due date}
%       \PURPOSE{Description of purpose}
%       \SUMMARY{Very short overview of assignment}
%       \DETAILS{Detailed description}
%         \SUBHEAD{if needed} ...
%         \SUBHEAD{if needed} ...
%          ...
%       \HANDIN{What to hand in and how}
%       \begin{checklist}
%       \item ...
%       \end{checklist}
% There is no need to include a "\documentstyle."
% However, there should be an "\end{document}."
%
%===========================================================
\documentclass[11pt,twoside,titlepage]{article}
%%NEED TO ADD epsf!!
\usepackage{threeparttop}
\usepackage{graphicx}
\usepackage{latexsym}
\usepackage{color}
\usepackage{listings}
\usepackage{fancyvrb}
%\usepackage{pgf,pgfarrows,pgfnodes,pgfautomata,pgfheaps,pgfshade}
\usepackage{tikz}
\usepackage[normalem]{ulem}
\tikzset{
    %Define standard arrow tip
%    >=stealth',
    %Define style for boxes
    oval/.style={
           rectangle,
           rounded corners,
           draw=black, very thick,
           text width=6.5em,
           minimum height=2em,
           text centered},
    % Define arrow style
    arr/.style={
           ->,
           thick,
           shorten <=2pt,
           shorten >=2pt,}
}
\usepackage[noend]{algorithmic}
\usepackage[noend]{algorithm}
\newcommand{\bfor}{{\bf for\ }}
\newcommand{\bthen}{{\bf then\ }}
\newcommand{\bwhile}{{\bf while\ }}
\newcommand{\btrue}{{\bf true\ }}
\newcommand{\bfalse}{{\bf false\ }}
\newcommand{\bto}{{\bf to\ }}
\newcommand{\bdo}{{\bf do\ }}
\newcommand{\bif}{{\bf if\ }}
\newcommand{\belse}{{\bf else\ }}
\newcommand{\band}{{\bf and\ }}
\newcommand{\breturn}{{\bf return\ }}
\newcommand{\mod}{{\rm mod}}
\renewcommand{\algorithmiccomment}[1]{$\rhd$ #1}
\newenvironment{checklist}{\par\noindent\hspace{-.25in}{\bf Checklist:}\renewcommand{\labelitemi}{$\Box$}%
\begin{itemize}}{\end{itemize}}
\pagestyle{threepartheadings}
\usepackage{url}
\usepackage{wrapfig}
% removing the standard hyperref to avoid the horrible boxes
%\usepackage{hyperref}
\usepackage[hidelinks]{hyperref}
% added in the dtklogos for the bibtex formatting
\usepackage{dtklogos}
%=========================
% One-inch margins everywhere
%=========================
\setlength{\topmargin}{0in}
\setlength{\textheight}{8.5in}
\setlength{\oddsidemargin}{0in}
\setlength{\evensidemargin}{0in}
\setlength{\textwidth}{6.5in}
%===============================
%===============================
% Macro for document title:
%===============================
\newcommand{\MYTITLE}[1]%
   {\begin{center}
     \begin{center}
     \bf
     CMPSC 280\\Principles of Software Development\\
     Fall 2015
     \medskip
     \end{center}
     \bf
     #1
     \end{center}
}
%================================
% Macro for headings:
%================================
\newcommand{\MYHEADERS}[2]%
   {\lhead{#1}
    \rhead{#2}
    %\immediate\write16{}
    %\immediate\write16{DATE OF HANDOUT?}
    %\read16 to \dateofhandout
    \def \dateofhandout {September 18, 2015}
    \lfoot{\sc Handed out on \dateofhandout}
    %\immediate\write16{}
    %\immediate\write16{HANDOUT NUMBER?}
    %\read16 to\handoutnum
    \def \handoutnum {5}
    \rfoot{Handout \handoutnum}
   }

%================================
% Macro for bold italic:
%================================
\newcommand{\bit}[1]{{\textit{\textbf{#1}}}}

%=========================
% Non-zero paragraph skips.
%=========================
\setlength{\parskip}{1ex}

%=========================
% Create various environments:
%=========================
\newcommand{\PURPOSE}{\par\noindent\hspace{-.25in}{\bf Purpose:\ }}
\newcommand{\SUMMARY}{\par\noindent\hspace{-.25in}{\bf Summary:\ }}
\newcommand{\DETAILS}{\par\noindent\hspace{-.25in}{\bf Details:\ }}
\newcommand{\HANDIN}{\par\noindent\hspace{-.25in}{\bf Hand in:\ }}
\newcommand{\SUBHEAD}[1]{\bigskip\par\noindent\hspace{-.1in}{\sc #1}\\}
%\newenvironment{CHECKLIST}{\begin{itemize}}{\end{itemize}}


\usepackage[compact]{titlesec}

\begin{document}
\MYTITLE{Laboratory Assignment Three: Team-Based Verification and Validation of Software}
\MYHEADERS{Laboratory Assignment Three}{Due: September 18, 2015}

\vspace*{-.1in}
\section*{Introduction}

Now that you are completely comfortable with the use of the Git version control repository, the Bitbucket site for managing
these Git repositories---and the use of the Markdown language for the purpose of documenting your software---we will
continue team-based implementation of a simple software system. In particular, your team will ``receive'' all of the
deliverables that another team implemented in the last laboratory assignment. Then, it will be your team's
responsibility to compare and contrast the deliverables that you created with those that the other team produced.

Next, you must carefully validate and verify both the deliverables that you received and those that you produced. For
this assignment, these tasks will involve you finding and discussing mistakes in the requirements, design, and tutorial
documents from your team and your ``partner'' team while also writing JUnit test cases for both the Java code that you
implemented and the code created by your team. Members of your team who have questions about the deliverables that you
received should direct them all to the maintainer from the team who shared their system with you.

\section*{Evaluating the Specification and Design of the Systems}

Paying careful attention to both the phases of the software lifecycle and the roles of members of a development team, as
explained in Sections 1.5 through 1.8 of the textbook, your team will interact with the customer to revisit the
requirements of the program and then, as needed, re-design, re-implement, re-document, and fully test the system.  The
maintainer of your system is also still responsible for interacting with the developers in your partner team who will
want to validate and verify your product. Teams with questions about these tasks should contact the course instructor.

Before you start working on this assignment, please carefully review the content at the end of Section 2.2 in SETP to
learn more about the tasks associated with the validation and verification of software. Once all members of your team
understand these tasks, you should pick one member who will be responsible for evaluating both the written deliverables
that you have received from your partner team and the system that you have created as part of the last assignment. At a
high level, you should begin to ask and objectively answer questions such as ``what are the similarities and differences
between our two systems?'' and ``which system is better and why do I think that it is better?'' and ``are these two
systems fundamentally similar to or different from each other?''

Specifically, this member of the team should assess the requirements documents in an effort to determine if it is
clearly written in a style that is precise, unambiguous, and free from errors in spelling and grammar. When this
individual is reviewing the object-oriented design, it is important to fully answer relevant questions such as: ``How
many classes does this design use?'' and ``What will be the relationship between the classes?'' and ``What methods do
the classes have?'' and ``What are the inputs and outputs of the methods?'' and ``Is the design easy to understand?''
Again, this team member should ask and answer all of these questions about the design of both their own system and the
one that was shared with them at the start of the laboratory assignment.

\section*{Verifying the Systems Through Testing}

In the last assignment, the developer(s) and tester(s) took the requirements and design documents and thought about how
the system should be implemented. For this assignment, one of the people who was not chosen to evaluate the requirements
and design documents should be tasked with validating the implementation of their own system. In particular, this task
involves writing complete JUnit test cases for all of the methods in your program. The other member of your team is
then responsible for writing tests in JUnit for the system that was shared with your team by the partner team. Whenever
possible, these two team members should try to re-use tests across the two systems.

The task of these two testers is to ensure that the program faithfully adheres to the description(s) already produced by
the specifiers and designers of both systems. When an aspect of the requirements and design documents is not clear, the
developer(s) must talk with maintainer of the system to resolve any outstanding concerns. If it is necessary to do so,
this laboratory assignment may necessitate that the requirements, design, or implementation of either system evolve so
as to best support comprehensive verification.  The creators of these deliverables must quickly commit any changes to
their repository so as to best ensure that the requirements and design of the system are in sync with its
implementation and the testing effort that you complete during this assignment.

At the start of this laboratory assignment, the two testers on your team may think that ``they do not have any work to
do''. First, these individuals should study sources on the Internet and in the ACM Digital Library to learn how to test
a Java program using JUnit test cases. Next, these two team members should review the content in Section 3.1 so that they
can learn more about project planning and the best way to define project milestones and activities. Using the knowledge
from this content in SETP, the team members should outline the activities and milestones that are relevant for the
completion of this assignment in one week. Finally, they can examine the content in Section 3.2 to learn more about the
roles in a software engineering project and the working styles that team members may exhibit during this assignment.
Ultimately, these team members must plan the completion of the project to ensure the timely delivery of two well-tested
programs.

% \section*{Carefully Review the Honor Code}

% The Academic Honor Program that governs the entire academic program at Allegheny College is described in the Allegheny
% Academic Bulletin.  The Honor Program applies to all work that is submitted for academic credit or to meet non-credit
% requirements for graduation at Allegheny College.  This includes all work assigned for this class (e.g., examinations,
%   laboratory assignments, and the final project).  All students who have enrolled in the College will work under the Honor
% Program.  Each student who has matriculated at the College has acknowledged the following pledge:

% \vspace*{-.05in}
% \begin{quote}
%   I hereby recognize and pledge to fulfill my responsibilities, as defined in the Honor Code, and to maintain the
%   integrity of both myself and the College community as a whole.
% \end{quote}
% \vspace*{-.05in}

% \noindent It is understood that an important part of the learning process in any course, and particularly one in
% computer science, derives from thoughtful discussions with teachers and fellow students.  Such dialogue is encouraged.
% However, it is necessary to distinguish carefully between the student who discusses the principles underlying a problem
% with others and the student who produces assignments that are identical to, or merely variations on, someone else's
% work.  While it is acceptable for students in this class to discuss their deliverables with their classmates, items that
% are nearly identical to the work of others will be taken as evidence of violating the Honor Code. In particular, make
% sure that your team members do not inappropriately communicate with the members of \mbox{other teams}.

\section*{Summary of the Required Deliverables}

This assignment invites you to submit printed and signed versions of the following deliverables; please see the
instructor if you have questions about any of these items. You must make sure that your team uses the same version
control repository in Bitbucket as you did for the last assignment when storing the evaluations and tests for your
project.  Also, you should store the evaluations and tests for your ``partner'' team's system in their shared
repository.  Don't forget that all of the documents must be prepared with Markdown and, for submission, converted to PDF
using {\tt pandoc}.

\vspace*{-.1in}
\begin{enumerate}
  \setlength{\itemsep}{0in}
  \item A complete description of the roles that each team member fulfilled.
  \item A full-featured evaluation of your system and the system that you received.
  \item A detailed description of the project management plan that your team followed.
  \item Well-documented Java source code for all of the tests for both of the systems.
  \item A tutorial that explains how to run the test suites for both of the systems.
  \item A collaboratively written ``lessons learned'' document reflecting on these two labs.
\end{enumerate}
\vspace*{-.1in}

\end{document}
