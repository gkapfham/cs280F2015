% Typical usage (all UPPERCASE items are optional):
%       \input 580pre
%       \begin{document}
%       \MYTITLE{Title of document, e.g., Lab 1\\Due ...}
%       \MYHEADERS{short title}{other running head, e.g., due date}
%       \PURPOSE{Description of purpose}
%       \SUMMARY{Very short overview of assignment}
%       \DETAILS{Detailed description}
%         \SUBHEAD{if needed} ...
%         \SUBHEAD{if needed} ...
%          ...
%       \HANDIN{What to hand in and how}
%       \begin{checklist}
%       \item ...
%       \end{checklist}
% There is no need to include a "\documentstyle."
% However, there should be an "\end{document}."
%
%===========================================================
\documentclass[11pt,twoside,titlepage]{article}
%%NEED TO ADD epsf!!
\usepackage{threeparttop}
\usepackage{graphicx}
\usepackage{latexsym}
\usepackage{color}
\usepackage{listings}
\usepackage{fancyvrb}
%\usepackage{pgf,pgfarrows,pgfnodes,pgfautomata,pgfheaps,pgfshade}
\usepackage{tikz}
\usepackage[normalem]{ulem}
\tikzset{
    %Define standard arrow tip
%    >=stealth',
    %Define style for boxes
    oval/.style={
           rectangle,
           rounded corners,
           draw=black, very thick,
           text width=6.5em,
           minimum height=2em,
           text centered},
    % Define arrow style
    arr/.style={
           ->,
           thick,
           shorten <=2pt,
           shorten >=2pt,}
}
\usepackage[noend]{algorithmic}
\usepackage[noend]{algorithm}
\newcommand{\bfor}{{\bf for\ }}
\newcommand{\bthen}{{\bf then\ }}
\newcommand{\bwhile}{{\bf while\ }}
\newcommand{\btrue}{{\bf true\ }}
\newcommand{\bfalse}{{\bf false\ }}
\newcommand{\bto}{{\bf to\ }}
\newcommand{\bdo}{{\bf do\ }}
\newcommand{\bif}{{\bf if\ }}
\newcommand{\belse}{{\bf else\ }}
\newcommand{\band}{{\bf and\ }}
\newcommand{\breturn}{{\bf return\ }}
\newcommand{\mod}{{\rm mod}}
\renewcommand{\algorithmiccomment}[1]{$\rhd$ #1}
\newenvironment{checklist}{\par\noindent\hspace{-.25in}{\bf Checklist:}\renewcommand{\labelitemi}{$\Box$}%
\begin{itemize}}{\end{itemize}}
\pagestyle{threepartheadings}
\usepackage{url}
\usepackage{wrapfig}
% removing the standard hyperref to avoid the horrible boxes
%\usepackage{hyperref}
\usepackage[hidelinks]{hyperref}
% added in the dtklogos for the bibtex formatting
\usepackage{dtklogos}
%=========================
% One-inch margins everywhere
%=========================
\setlength{\topmargin}{0in}
\setlength{\textheight}{8.5in}
\setlength{\oddsidemargin}{0in}
\setlength{\evensidemargin}{0in}
\setlength{\textwidth}{6.5in}
%===============================
%===============================
% Macro for document title:
%===============================
\newcommand{\MYTITLE}[1]%
   {\begin{center}
     \begin{center}
     \bf
     CMPSC 280\\Principles of Software Development\\
     Fall 2015
     \medskip
     \end{center}
     \bf
     #1
     \end{center}
}
%================================
% Macro for headings:
%================================
\newcommand{\MYHEADERS}[2]%
   {\lhead{#1}
    \rhead{#2}
    %\immediate\write16{}
    %\immediate\write16{DATE OF HANDOUT?}
    %\read16 to \dateofhandout
    \def \dateofhandout {September 18, 2015}
    \lfoot{\sc Handed out on \dateofhandout}
    %\immediate\write16{}
    %\immediate\write16{HANDOUT NUMBER?}
    %\read16 to\handoutnum
    \def \handoutnum {5}
    \rfoot{Handout \handoutnum}
   }

%================================
% Macro for bold italic:
%================================
\newcommand{\bit}[1]{{\textit{\textbf{#1}}}}

%=========================
% Non-zero paragraph skips.
%=========================
\setlength{\parskip}{1ex}

%=========================
% Create various environments:
%=========================
\newcommand{\PURPOSE}{\par\noindent\hspace{-.25in}{\bf Purpose:\ }}
\newcommand{\SUMMARY}{\par\noindent\hspace{-.25in}{\bf Summary:\ }}
\newcommand{\DETAILS}{\par\noindent\hspace{-.25in}{\bf Details:\ }}
\newcommand{\HANDIN}{\par\noindent\hspace{-.25in}{\bf Hand in:\ }}
\newcommand{\SUBHEAD}[1]{\bigskip\par\noindent\hspace{-.1in}{\sc #1}\\}
%\newenvironment{CHECKLIST}{\begin{itemize}}{\end{itemize}}


\usepackage[compact]{titlesec}

\begin{document}
\MYTITLE{Laboratory Assignment Five: \\ Releasing and Maintaining a Programming Systems Product}
\MYHEADERS{Laboratory Assignment Five}{Due: October 2, 2015}

\vspace*{-.1in}
\section*{Introduction}

In the past laboratory assignment, your team of three individuals created a preliminary version of a programming systems
product and prepared a short presentation describing your system. At the start of today's laboratory session, you will
hear the presentations from each of the teams and then decide which of these systems are best suited to moving forward
for a public release as part of the current assignment. The ultimate goal for this laboratory assignment is for you and
your team to create a publicly available GitHub repository that contains a complete, full-featured, working, and
properly documented programming systems product. As part of the completion of this assignment, your team will use
features of GitHub to, for instance, resolve issues raised by people who have started to use your software tool. During
the start of next week's laboratory session, you will give a detailed presentation describing all aspects of your system
and then participate in a release party.

\section*{Organizing Your Software Development Team}

Please organize yourselves into teams of exactly six students. All of the students in this class should work together to
identify which of the six projects are most likely to yield a high-quality system that will be of practical use to
individuals who have many Git repositories in their filesystem. Then, you should select three of the six projects and
organize yourselves into teams of six students each.  The membership of the new teams must consist of entire teams from
the previous assignment. That is, all of the members of one team must agree to join together with all of the members of
another team.  As you are forming your teams, please think about the strengths and weaknesses of the potential members
of your development team. Next, each of your teams should pick one person to serve in the role of ``chief programmer''
(CP) --- this individual will lead the effort to release your system to GitHub and ensure the conceptual integrity of
your programming systems product.

Since the purpose of this assignment is to release a final product to GitHub, it is important to ensure that you
assemble a team with a diverse set of skills. While it is important to have some strong programmers on your team since
you may add and/or enhance features of your software tool, it is also crucial for you to have good writers and
presenters who can take the lead in, for example, polishing the requirements and design documents, revising and
extending the tutorial, and creating a lengthy presentation.  Finally, your team should ensure that all of your members
know how to effectively use Git and Slack and then make a plan for how you will control your source code and
documentation and communication using channel messages and integrations.

At this start of this assignment, each member of your team should create a GitHub account and add all of their relevant
details to their account's profile (i.e., each student should include their full name, Web site, and profile picture in
their GitHub account).  Next, your team should pick a name for your project that best reflects the features that you
intend to have in the final version of your tool. Now, your team's CP should create a GitHub repository (that has your
chosen name) and ensure that each team member is a ``collaborator'' who has write access. Additionally, the CP must
ensure that the course instructor has write access to the GitHub repository. When your team is transitioning from the
use of Bitbucket to GitHub for this assignment, you may consider adding the new GitHub repository as a ``remote'' for an
existing Bitbucket repository. Please see the course instructor if you have questions about creating and configuring
your GitHub repository.

\section*{Clarifying the Customer's Requirements}

Once you have formed your team and created and configured your new GitHub repository, then you and your team should
decide what features will be part of the final programming systems product. Since your team will have members from the
smaller teams of the past laboratory assignment, the CP should ensure that all potential features are openly discussed
--- that is, it may be possible for the final version of your tool to have features from the projects of several
different members of your new team. Next, your team should identify individual(s) who can continue to clarify the
customer's requirements for the tool. In particular, if your team develops ideas for new features, then it is crucial
for you to present these to the customer before starting implementation.

Using the requirements documents produced by your past teams as a starting point, your new team should prepare a final
requirements document that clearly explains all of the functional (behavior) and non-functional (quality) requirements
for your system. Written in Markdown, your final requirements document should be available in your project's GitHub
repository. As you are finishing your requirements document, please ask yourself the following questions as you
determine whether or not you have written high-quality and useful requirements. Students who want to learn more about
the correct way to state the requirements of software system should review Sections 4.1 through 4.4 of SETP, paying
particular attention to the description different types of requirements.

\vspace*{-.05in}
\begin{enumerate}

  \itemsep 0in

  \item Are the requirements correct?
  \item Are the requirements consistent?
  \item Are the requirements unambiguous?
  \item Are the requirements complete?
  \item Are the requirements feasible?
  \item Is every requirement relevant?
  \item Are the requirements testable?
  \item Are the requirements traceable?

\end{enumerate}

\section*{Improving the System's Design}

Using the updated requirements document, your team should identify individual(s) who can translate the revised
requirements document into a complete design that encompasses all of the system's intended features. Your final design
document, including any technical diagrams that illustrate the system's design, should be made publicly available in
your team's GitHub repository.

As you are finalizing the system's design, you should try to further develop answers to relevant questions such as: How
many components will you use? What will be the relationship between the components? What functions will the components
have?  What will be the inputs and outputs of the functions?  Is the design easy to understand?  Can you actually
implement and test this design? The answers to these and other relevant questions should be written in a Markdown-based
design document containing text and diagrams that explain the system. Since the main focus of this assignment is not
design diagrams, students may use any reasonable tool to create their diagrams and, if necessary, include them as
separate files in their team's repository.  When clarification of the system's specification is necessary (e.g., due to
writing ambiguities), the designers should interact with the individual(s) who elicited \mbox{the system's
requirements}.  Ultimately, your team should carefully ensure that the system's requirements and design are correctly in
sync.

\section*{Implementing and Testing the System}

The developer(s) and tester(s) on your team should take the revised requirements and design documents and start to think
about how the system will be (re-)implemented. Your task is to ensure that the program faithfully adheres to the descriptions
already produced by other members of your team. When an aspect of the requirements and design documents is not clear,
the developer(s) must talk with the people who created these documents to resolve any outstanding concerns. The creators
of these documents must quickly commit any changes in them to their repository so as to best ensure that the
requirements and design of the system are in sync with its implementation. Please bear in mind that it is now acceptable
to share source code that was created by any member of the two teams that were combined to form the larger team for this
assignment.

As this programming systems product will be initially released to GitHub by Monday, September 28, 2015, the implementors
and testers should take care to create a system that is well-documented through comments and other relevant
documentation. Whenever it is possible to do so, these members of your team must also add (semi-)automated methods for
verifying that the implementation adheres to the requirements that previously were elicited from the customer. For
instance, if your team decides to implement {\tt git-beagle} in the Java programming language, then you should test it
with automated tests cases that you wrote in JUnit, the industry standard for automated testing in Java. Overall, as you
are implementing and testing your system you should hold yourselves to a high standard under the assumption that other
software engineers will review and use your code.

\section*{Ensuring the Effective Operation of Your Team}

When you start to work on this laboratory assignment, it may seem as though the designer(s), developer(s), and tester(s)
``do not have any work to do'' because the requirements of the system have not been established. Yet, if you carefully
think about the work that you must complete for this assignment, it will become clear that this is not the case! For
instance, one member of your team should be tasked with creating all of the needed means for communication with tools
such as Slack and Bitbucket. Additionally, team members who are waiting to complete their chosen tasks should consider
investigating the use of tools, like Trello, to organize their team's efforts. Finally, it is important for team
members to spend time creating the templates for their deliverables and then carefully ``sharpening their tools''. For
instance, the developer(s) on the team can ensure that they have a smoothly functioning development environment that
will support the implementation of well-documented and correct code. Please see the instructor if you have questions
about \mbox{this matter}.

\section*{Presenting Your Programming Systems Product}

At the start of next week's laboratory session, each team will give a short five minute presentation explaining their
own implementation of the {\tt git-beagle} concept. You will be responsible for highlighting the key features of your
tool and the ways in which you did your best to specify, design, implement, release, and document it. Whenever possible,
you should give a high-quality, interactive, and engaging presentation that is both fun, interesting, and technically
correct. Unless a team can demonstrate that it is not possible for them to implement their presentation as a program,
all teams should use frameworks like {\tt reveal.js}, {\tt big}, or {\tt beamer} to create their presentations. Creating
your presentations using one of these tools is ideal because it will allow your team to store all of its content in a
Git version control repository and thus ultimately be released with the rest of your programming systems product.
Students who would like to learn more about implementing presentations with one of these frameworks can study some of
the examples that are available in the course instructor's GitHub repositories available at {\tt
https://github.com/gkapfham/}.

\section*{Carefully Review the Honor Code}

The Academic Honor Program that governs the entire academic program at Allegheny College is described in the Allegheny
Academic Bulletin.  The Honor Program applies to all work that is submitted for academic credit or to meet non-credit
requirements for graduation at Allegheny College.  This includes all work assigned for this class (e.g., examinations,
  laboratory assignments, and the final project).  All students who have enrolled in the College will work under the Honor
Program.  Each student who has matriculated at the College has acknowledged the following pledge:

\vspace*{-.05in}
\begin{quote}
  I hereby recognize and pledge to fulfill my responsibilities, as defined in the Honor Code, and to maintain the
  integrity of both myself and the College community as a whole.
\end{quote}
\vspace*{-.05in}

\noindent It is understood that an important part of the learning process in any course, and particularly one in
computer science, derives from thoughtful discussions with teachers and fellow students.  Such dialogue is encouraged.
However, it is necessary to distinguish carefully between the student who discusses the principles underlying a problem
with others and the student who produces assignments that are identical to, or merely variations on, someone else's
work.  While it is acceptable for students in this class to discuss their deliverables with their classmates, items that
are nearly identical to the work of others will be taken as evidence of violating the Honor Code. In particular, make
sure that your team members do not inappropriately share deliverables with the members of \mbox{other teams}.

\section*{Summary of the Required Deliverables}

This assignment invites you to submit printed and signed versions of the following deliverables; please see the
instructor if you have questions about any of these items. Please make sure that your team creates a version control
repository in Bitbucket to store all of the deliverables for this assignment; since you will again select your own
teams, please create a repository with a numerically-based name and share it with the course instructor. Also note that
all of the written documents must be prepared with Markdown and, for submission, converted to PDF using {\tt pandoc}.

\vspace*{-.1in}
\begin{enumerate}
  \setlength{\itemsep}{0in}
  \item A complete description of the roles that each team member fulfilled.
  \item A requirements document that fully describes the features of your system.
  \item Expressed in writing and as a simple technical diagram, the design of your system.
  \item Well-documented source code that fulfills all of the system's requirements.
  \item A tutorial that explains how to use all of the features of your finished system.
  \item Individually completed and submitted reviews of all of the team members.
  \item An informative and interesting five-minute presentation that highlights your software.
\end{enumerate}
\vspace*{-.1in}

\end{document}
