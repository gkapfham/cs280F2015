% Typical usage (all UPPERCASE items are optional):
%       \input 580pre
%       \begin{document}
%       \MYTITLE{Title of document, e.g., Lab 1\\Due ...}
%       \MYHEADERS{short title}{other running head, e.g., due date}
%       \PURPOSE{Description of purpose}
%       \SUMMARY{Very short overview of assignment}
%       \DETAILS{Detailed description}
%         \SUBHEAD{if needed} ...
%         \SUBHEAD{if needed} ...
%          ...
%       \HANDIN{What to hand in and how}
%       \begin{checklist}
%       \item ...
%       \end{checklist}
% There is no need to include a "\documentstyle."
% However, there should be an "\end{document}."
%
%===========================================================
\documentclass[11pt,twoside,titlepage]{article}
%%NEED TO ADD epsf!!
\usepackage{threeparttop}
\usepackage{graphicx}
\usepackage{latexsym}
\usepackage{color}
\usepackage{listings}
\usepackage{fancyvrb}
%\usepackage{pgf,pgfarrows,pgfnodes,pgfautomata,pgfheaps,pgfshade}
\usepackage{tikz}
\usepackage[normalem]{ulem}
\tikzset{
    %Define standard arrow tip
%    >=stealth',
    %Define style for boxes
    oval/.style={
           rectangle,
           rounded corners,
           draw=black, very thick,
           text width=6.5em,
           minimum height=2em,
           text centered},
    % Define arrow style
    arr/.style={
           ->,
           thick,
           shorten <=2pt,
           shorten >=2pt,}
}
\usepackage[noend]{algorithmic}
\usepackage[noend]{algorithm}
\newcommand{\bfor}{{\bf for\ }}
\newcommand{\bthen}{{\bf then\ }}
\newcommand{\bwhile}{{\bf while\ }}
\newcommand{\btrue}{{\bf true\ }}
\newcommand{\bfalse}{{\bf false\ }}
\newcommand{\bto}{{\bf to\ }}
\newcommand{\bdo}{{\bf do\ }}
\newcommand{\bif}{{\bf if\ }}
\newcommand{\belse}{{\bf else\ }}
\newcommand{\band}{{\bf and\ }}
\newcommand{\breturn}{{\bf return\ }}
\newcommand{\mod}{{\rm mod}}
\renewcommand{\algorithmiccomment}[1]{$\rhd$ #1}
\newenvironment{checklist}{\par\noindent\hspace{-.25in}{\bf Checklist:}\renewcommand{\labelitemi}{$\Box$}%
\begin{itemize}}{\end{itemize}}
\pagestyle{threepartheadings}
\usepackage{url}
\usepackage{wrapfig}
% removing the standard hyperref to avoid the horrible boxes
%\usepackage{hyperref}
\usepackage[hidelinks]{hyperref}
% added in the dtklogos for the bibtex formatting
\usepackage{dtklogos}
%=========================
% One-inch margins everywhere
%=========================
\setlength{\topmargin}{0in}
\setlength{\textheight}{8.5in}
\setlength{\oddsidemargin}{0in}
\setlength{\evensidemargin}{0in}
\setlength{\textwidth}{6.5in}
%===============================
%===============================
% Macro for document title:
%===============================
\newcommand{\MYTITLE}[1]%
   {\begin{center}
     \begin{center}
     \bf
     CMPSC 280\\Principles of Software Development\\
     Fall 2015
     \medskip
     \end{center}
     \bf
     #1
     \end{center}
}
%================================
% Macro for headings:
%================================
\newcommand{\MYHEADERS}[2]%
   {\lhead{#1}
    \rhead{#2}
    %\immediate\write16{}
    %\immediate\write16{DATE OF HANDOUT?}
    %\read16 to \dateofhandout
    \def \dateofhandout {September 18, 2015}
    \lfoot{\sc Handed out on \dateofhandout}
    %\immediate\write16{}
    %\immediate\write16{HANDOUT NUMBER?}
    %\read16 to\handoutnum
    \def \handoutnum {5}
    \rfoot{Handout \handoutnum}
   }

%================================
% Macro for bold italic:
%================================
\newcommand{\bit}[1]{{\textit{\textbf{#1}}}}

%=========================
% Non-zero paragraph skips.
%=========================
\setlength{\parskip}{1ex}

%=========================
% Create various environments:
%=========================
\newcommand{\PURPOSE}{\par\noindent\hspace{-.25in}{\bf Purpose:\ }}
\newcommand{\SUMMARY}{\par\noindent\hspace{-.25in}{\bf Summary:\ }}
\newcommand{\DETAILS}{\par\noindent\hspace{-.25in}{\bf Details:\ }}
\newcommand{\HANDIN}{\par\noindent\hspace{-.25in}{\bf Hand in:\ }}
\newcommand{\SUBHEAD}[1]{\bigskip\par\noindent\hspace{-.1in}{\sc #1}\\}
%\newenvironment{CHECKLIST}{\begin{itemize}}{\end{itemize}}


\usepackage[compact]{titlesec}

\usepackage[url=false,
    backend=biber,
    style=authoryear,
    doi=false,
    isbn=false,
    backref=false,
    dashed=false,                                   % Do not add a dash out authors for subsequent articles with the same authors.
    maxnames=99,                                    % Amount of authors to include before abbreviating.
    sorting=ydnt]{biblatex}                         % Sorting in reverse order
\renewbibmacro{in:}{}

\addbibresource{bibliography_schemaanalyst.bib}

\begin{document}
\MYTITLE{Laboratory Assignment Eight: Understanding and Releasing Real-World Software}
\MYHEADERS{Laboratory Assignment Eight}{Due: November 13, 2015}

\vspace*{-.1in}
\section*{Introduction}

\nocite{*}

Throughout the semester, you have learned about the phases of the software development lifecycle and both the software
tools and the non-cognitive skills that are needed to support the creation of high-quality software in each of these
phases. Yet, none of the past laboratory assignments required you to deliver a programming systems product to an
external customer. In this laboratory assignment, you will gain experience in working with and understanding a large,
real-world software system created by an external customer who wants to release it to GitHub under a free and
open-source license. Additionally, you will use tools, like JDepend and JavaNCSS, to characterize to a large existing
system. You and your team members will also explore both the issues related to effective technology transfer and the
ways to improve the quality of a software system's documentation. Finally, you will learn more about the licensing of
free and open-source software.

\vspace*{-.05in}
\section*{Learning About SchemaAnalyst}

Many real-world software systems interact with a relational database management system (RDBMS) that hosts a database
used for persisting the program's data. The database itself is protected by a schema that governs the types of data
values that are allowed into the database. If the schema of the database is not correctly defined, then the database may
either become corrupted with incorrect values or it may reject values that should be allowed into the storage system.
The SchemaAnalyst tool, implemented primarily by Phil McMinn and Chris J. Wright, automatically generates test cases
that help a database designer establish a confidence in the correctness of the database.

Before organizing your teams and starting to learn more about the fundamental ideas underlying SchemaAnalyst's approach
to testing database schemas. You can do this by reading some of the research papers --- listed in the ``References''
section of this assignment --- that McMinn, Wright, and Kapfhammer published about SchemaAnalyst. While it is not
necessary for you to understand all of the technical and mathematical details in these papers, a familiarity with the
concepts developed by McMinn and Wright will help your team to successfully complete all of the deliverables required
for this project. For instance, an understanding of the basics about SchemaAnalyst will allow you to writing good
introductory content for the tool's {\tt README.md} file on its GitHub site. Please see the instructor if you have
questions about any of these papers.

\vspace*{-.05in}
\section*{Accessing and Using SchemaAnalyst}

Currently, the source code of SchemaAnalyst is stored in a private Git repository hosted by BitBucket. Your team should
pick a leader who will then ask the instructor for access to this repository, create a private BitBucket repository for
your team to complete this project, and then move all of SchemaAnalyst's source code to your team's private repository.
Please make sure that you share your team's repository with the instructor and both Phil McMinn and Chris J.
Wright, your external customers for this assignment. In addition, your customers have created a preliminary draft of
user documentation for SchemaAnalyst and provided this to you in a separate Git repository; please have your team's
leader ask the instructor for access to this repository and then use this documentation as a start for the {\tt
README.md} file that will appear on SchemaAnalyst's GitHub page.

Next, each member of your team should review the source code of SchemaAnalyst's build system provided in the {\tt
build.xml} file. Can you learn how to use this system to compile the SchemaAnalyst system? While compiling can you find
out how many Java classes are inside of SchemaAnalyst? Once you have successfully compiled SchemaAnalyst, you should
study the user documentation to learn more about how to run the different parts of this tool. Can each member of your
team use SchemaAnalyst to generate a test suite for a sample schema of a relational database? What are the inputs and
outputs of SchemaAnalyst when it generates test data?

Of course, it is also important to characterize the quality of a test suite that SchemaAnalyst automatically generates.
One way to evaluate tests is to use mutation analysis; please refer to the papers in the ``References'' section to learn
more about how mutation analysis can help you to understand the quality of a test suite. Can you and your team members
run the mutation analysis feature provided by SchemaAnalyst? What are the inputs and outputs of SchemaAnalyst when it
performs a mutation analysis of a test suite?

\section*{Preparing SchemaAnalyst for Public Release}

\section*{Evaluating SchemaAnalyst's Design and Implementation}

\section*{Improving the Programming Systems Product}

\section*{Review the Textbook for Key Ideas}

To best ensure your team's success at completing this project, you should review all of the content in Chapters 1--6 of
SETP, focusing on Chapter 2's relevant details about the process of developing software. Every member of the team should
also review Chapters 1--4 of MMM, reconsidering topics such as the roles in a software development team and the
techniques that help developers achieve conceptual integrity in a design. Refreshing your understanding of this content
will best ensure that your team works well together when completing this assignment --- and help you to prepare for the
final project where you will deliver a large-scale system to an external customer.

Although not absolutely required for the successful completion of this laboratory assignment, you may also wish to
review SETP's material about software documentation (see Section 10.2) and technology transfer (see Section 14.2).
Additionally, you may examine Chapter 15 of MMM to learn more about ways in which you can document software. Although
this optional content in SETP and MMM reinforces concepts already introduced in past class and laboratory sessions, it
may provide further context to support your release a programming systems product on behalf of an external customer.
Please see the instructor if you have questions about any of these \mbox{reading assignments}.

\section*{Presentation of Important Insights}

\section*{Summary of the Required Deliverables}

This assignment invites your team to submit one printed version of the following files:
\vspace*{-.1in}
\begin{enumerate}
    \itemsep0em
    \item A description of and justification for your team's chosen organization, roles, and tool support
    \item A document that clearly explains the meaning of JDepend's design quality metrics
    \item A document that clearly explains the meaning of the metrics calculated by JavaNCSS
    \item An analysis of the values of all the relevant metrics for two previously implemented systems
    \item An analysis of the values of all the relevant metrics for the two open-source systems
    \item Modified implementation artifacts of your chosen systems (e.g., build system and source code)
    \item The slides of the presentation that you will give at the start of the next laboratory session
\end{enumerate}
\vspace*{-.1in}

\defbibfilter{papers}{
  type=article or
  type=inproceedings or
  type=incollection and
  not keyword=edit
}
\DeclareFieldFormat[article]{volume}{\addcomma\addspace#1\nopunct}
\DeclareFieldFormat[article]{number}{(#1)}
\DeclareFieldFormat[inproceedings]{booktitle}{in \em{#1}}
\DeclareFieldFormat[article]{title}{\mkbibquote{#1\addcomma}}
\DeclareFieldFormat[inproceedings]{title}{\mkbibquote{#1\addcomma}}
\DeclareFieldFormat[incollection]{title}{\mkbibquote{#1\addcomma}}
\printbibliography[filter=papers,title={References}]

\end{document}
