\input{labspre.tex}

\usepackage[compact]{titlesec}

\begin{document}
\MYTITLE{Laboratory Assignment Two: Starting with Team-Based Implementation of Software}
\MYHEADERS{Laboratory Assignment Two}{Due: September 11, 2015}

\section*{Introduction}

Now that you understand how to use the Git version control repository, the Bitbucket system for managing these Git
repositories, and the use of Markdown for the purpose of documenting software, we are going to start team-based
implementation of a simple software system. Paying careful attention to both the phases of the software lifecycle and
the roles of members of the development team, as explained in Sections 1.5 through 1.8 of the textbook, you should
interact with a customer to learn the requirements of a program and then design, implement, document, and release the
system. Your team of software engineers should also select one individual who will serve as the maintainer of the system
and thus be responsible for interacting with any future customers or developers who will want to use or extend your
product.

\section*{Configuring Git and Bitbucket}

\section*{Summary of the Required Deliverables}

This assignment invites you to submit printed and signed versions of the following deliverables; please see the
instructor if you have questions about these items. Also note that all of the written documents must be prepared with
Markdown and, for submission, converted to PDF using {\tt pandoc}.

\vspace*{-.1in}
\begin{enumerate}
  \setlength{\itemsep}{0in}
  \item A complete description of the roles that each team member fulfilled.
  \item A requirements document that describes the features of your system.
  \item Expressed as a simple technical diagram, a design of your system.
  \item Well-documented Java source code that implements the full system.
  \item A tutorial that explains how to use all of the features of your system.
  \item Individually completed and submitted reviews of all of the team members.
\end{enumerate}
\vspace*{-.1in}

\end{document}
