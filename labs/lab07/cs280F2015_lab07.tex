% Typical usage (all UPPERCASE items are optional):
%       \input 580pre
%       \begin{document}
%       \MYTITLE{Title of document, e.g., Lab 1\\Due ...}
%       \MYHEADERS{short title}{other running head, e.g., due date}
%       \PURPOSE{Description of purpose}
%       \SUMMARY{Very short overview of assignment}
%       \DETAILS{Detailed description}
%         \SUBHEAD{if needed} ...
%         \SUBHEAD{if needed} ...
%          ...
%       \HANDIN{What to hand in and how}
%       \begin{checklist}
%       \item ...
%       \end{checklist}
% There is no need to include a "\documentstyle."
% However, there should be an "\end{document}."
%
%===========================================================
\documentclass[11pt,twoside,titlepage]{article}
%%NEED TO ADD epsf!!
\usepackage{threeparttop}
\usepackage{graphicx}
\usepackage{latexsym}
\usepackage{color}
\usepackage{listings}
\usepackage{fancyvrb}
%\usepackage{pgf,pgfarrows,pgfnodes,pgfautomata,pgfheaps,pgfshade}
\usepackage{tikz}
\usepackage[normalem]{ulem}
\tikzset{
    %Define standard arrow tip
%    >=stealth',
    %Define style for boxes
    oval/.style={
           rectangle,
           rounded corners,
           draw=black, very thick,
           text width=6.5em,
           minimum height=2em,
           text centered},
    % Define arrow style
    arr/.style={
           ->,
           thick,
           shorten <=2pt,
           shorten >=2pt,}
}
\usepackage[noend]{algorithmic}
\usepackage[noend]{algorithm}
\newcommand{\bfor}{{\bf for\ }}
\newcommand{\bthen}{{\bf then\ }}
\newcommand{\bwhile}{{\bf while\ }}
\newcommand{\btrue}{{\bf true\ }}
\newcommand{\bfalse}{{\bf false\ }}
\newcommand{\bto}{{\bf to\ }}
\newcommand{\bdo}{{\bf do\ }}
\newcommand{\bif}{{\bf if\ }}
\newcommand{\belse}{{\bf else\ }}
\newcommand{\band}{{\bf and\ }}
\newcommand{\breturn}{{\bf return\ }}
\newcommand{\mod}{{\rm mod}}
\renewcommand{\algorithmiccomment}[1]{$\rhd$ #1}
\newenvironment{checklist}{\par\noindent\hspace{-.25in}{\bf Checklist:}\renewcommand{\labelitemi}{$\Box$}%
\begin{itemize}}{\end{itemize}}
\pagestyle{threepartheadings}
\usepackage{url}
\usepackage{wrapfig}
% removing the standard hyperref to avoid the horrible boxes
%\usepackage{hyperref}
\usepackage[hidelinks]{hyperref}
% added in the dtklogos for the bibtex formatting
\usepackage{dtklogos}
%=========================
% One-inch margins everywhere
%=========================
\setlength{\topmargin}{0in}
\setlength{\textheight}{8.5in}
\setlength{\oddsidemargin}{0in}
\setlength{\evensidemargin}{0in}
\setlength{\textwidth}{6.5in}
%===============================
%===============================
% Macro for document title:
%===============================
\newcommand{\MYTITLE}[1]%
   {\begin{center}
     \begin{center}
     \bf
     CMPSC 280\\Principles of Software Development\\
     Fall 2015
     \medskip
     \end{center}
     \bf
     #1
     \end{center}
}
%================================
% Macro for headings:
%================================
\newcommand{\MYHEADERS}[2]%
   {\lhead{#1}
    \rhead{#2}
    %\immediate\write16{}
    %\immediate\write16{DATE OF HANDOUT?}
    %\read16 to \dateofhandout
    \def \dateofhandout {September 18, 2015}
    \lfoot{\sc Handed out on \dateofhandout}
    %\immediate\write16{}
    %\immediate\write16{HANDOUT NUMBER?}
    %\read16 to\handoutnum
    \def \handoutnum {5}
    \rfoot{Handout \handoutnum}
   }

%================================
% Macro for bold italic:
%================================
\newcommand{\bit}[1]{{\textit{\textbf{#1}}}}

%=========================
% Non-zero paragraph skips.
%=========================
\setlength{\parskip}{1ex}

%=========================
% Create various environments:
%=========================
\newcommand{\PURPOSE}{\par\noindent\hspace{-.25in}{\bf Purpose:\ }}
\newcommand{\SUMMARY}{\par\noindent\hspace{-.25in}{\bf Summary:\ }}
\newcommand{\DETAILS}{\par\noindent\hspace{-.25in}{\bf Details:\ }}
\newcommand{\HANDIN}{\par\noindent\hspace{-.25in}{\bf Hand in:\ }}
\newcommand{\SUBHEAD}[1]{\bigskip\par\noindent\hspace{-.1in}{\sc #1}\\}
%\newenvironment{CHECKLIST}{\begin{itemize}}{\end{itemize}}


\usepackage[compact]{titlesec}

\begin{document}
\MYTITLE{Laboratory Assignment Seven: Characterizing the Design of Programs and Tests}
\MYHEADERS{Laboratory Assignment Seven}{Due: October 30, 2015}

\section*{Introduction}

In the previous laboratory assignments, you have learned much of the knowledge and skills that we will use to specify,
design, implement, test, document, and release the final project for this course.  You have also had several experiences
with working in both small and ``large'' teams to complete the phases of the software life cycle, with a recent focus on
the elicitation of software requirements and the planning of software projects.  In this assignment, you and your team
will install, configure, and use software tools that automatically calculate design and implementation metrics that
characterize programs and test suites.  One week from now, you will present the findings from your analyses.

\section*{Calculating Design Quality Metrics with JDepend}

After you have written a requirements document for your system and completed the description of the architecture, you
must create a design for your program.  Of course, it is important to evaluate the quality of your design.  If you
already have a (partial) implementation of a program and you want to evaluate the quality of its design, you can use a
tool called JDepend.  You can learn more about JDepend by visiting \url{http://clarkware.com/software/JDepend.html}.
What are the design quality metrics that JDepend calculates? What are the meanings of these metrics? How can you use
these metrics to better understand and modify the design of a program? As one of the deliverables for this
assignment, your team should use \LaTeX\ or Markdown to prepare a document that contains formal definitions and
equations that describe all of the design quality metrics.

Once you and your team have investigated and discussed the features provided by JDepend, you should pick two separate
systems that have been previously implemented by members of your team. Now, copy the source code from these two projects
--- and any other files that your team deem to be relevant --- to the version control repository that you will use for
this project.  After downloading and installing JDepend, you should learn how to use it and then attempt to perform a
JDepend-based analysis of the project. What does the output of JDepend tell you about the quality of each project's design? Do
the chosen programs have good designs? Why or why not?

\section*{Source Code Measurement with JavaNCSS}

As you are programming your system, it is important to regularly calculate metrics that characterize the quality of your
implementation.  JavaNCSS is a software tool that can automatically scan your Java source code and report information
about the number of non-commented source code statements and the cyclomatic complexity.  You can learn more about
JavaNCSS by visiting \url{http://www.kclee.de/clemens/java/javancss/}. After reading the Web site for JavaNCSS, you
should search for papers in the ACM Digital Library, available at \url{http://dl.acm.org/}, that formally describe the
meaning of the metrics calculated by JavaNCSS, such as cyclomatic complexity. As one of the deliverables for this
assignment, your team should use \LaTeX\ or Markdown to write formal definitions and equations that explain all of the
metrics calculated by JavaNCSS.

Once all of the members of your team understand the metrics calculated by JavaNCSS, you should download and install this
tool.  As you did with JDepend, now you must learn how to run JavaNCSS and then apply it to each project's source code.
What does the output of JavaNCSS tell you about the quality of each project's implementation? Do your chosen programs
exhibit good implementation characteristics? How can you use the values of these metrics to better understand and modify
the implementation of your Java programs? How much of your system's source code is for the program itself? How much of
the source code is devoted to the test suite?

\section*{Analyzing Real-World Programs}

After your team finishes the analysis of two projects that you completed in previous laboratory assignments, you should
identify two open-source Java projects that you can download.  After picking the projects that you want to further
analyze, you can create separate directories for them in your Bitbucket repository for this laboratory assignment.  What
interesting trends can you find in the design and implementation quality metrics for the two chosen open-source
programs?  How do the values of these metrics for the open-source programs compare to the values for the projects that
you implemented in this class?  What can you learn about ways to improve the design and implementation of these Java
programs? You should answer these questions by using JDepend and JavaNCSS to analyze each of the four (in total)
programs.  Students who want to learn more about methods for architecting and designing a software system should review
the content in Chapters 5 and 6 of the SETP textbook. Please see the instructor if you have questions about this task.

\section*{Presentation of Important Insights}

Once you have completed all of the previous phases of this assignment, you should prepare a short ten minute
presentation that your entire team will give at the start of the next laboratory session. Leveraging one of the
HTML-based presentation formats that you used in previous laboratory assignments, you should prepare slides highlighting
some of the most important insights that you garnered from the automated analysis of the design and implementation of
your chosen Java programs.

\section*{Summary of the Required Deliverables}

This assignment invites your team to submit one printed version of the following files:
\vspace*{-.1in}
\begin{enumerate}
    \itemsep0em
    \item A description of and justification for your team's chosen organization, roles, and tool support
    \item A document that clearly explains the meaning of JDepend's design quality metrics
    \item A document that clearly explains the meaning of the metrics calculated by JavaNCSS
    \item An analysis of the values of all the relevant metrics for two previously implemented systems
    \item An analysis of the values of all the relevant metrics for the two open-source systems
    \item Modified implementation artifacts of your chosen systems (e.g., build system and source code)
    \item The slides of the presentation that you will give at the start of the next laboratory session
\end{enumerate}
\vspace*{-.1in}

You must also ensure that the instructor has read access to your Bitbucket repository that is named according to the
convention {\tt cs280F2015-lab7-team{\em k}}, with {\tt {\em k}} representing the number of your assigned team.  Your
repository should contain all of the deliverables that you produced during the completion of this assignment.  Please
see the course instructor if you have any questions!

\end{document}
