% Typical usage (all UPPERCASE items are optional):
%       \input 580pre
%       \begin{document}
%       \MYTITLE{Title of document, e.g., Lab 1\\Due ...}
%       \MYHEADERS{short title}{other running head, e.g., due date}
%       \PURPOSE{Description of purpose}
%       \SUMMARY{Very short overview of assignment}
%       \DETAILS{Detailed description}
%         \SUBHEAD{if needed} ...
%         \SUBHEAD{if needed} ...
%          ...
%       \HANDIN{What to hand in and how}
%       \begin{checklist}
%       \item ...
%       \end{checklist}
% There is no need to include a "\documentstyle."
% However, there should be an "\end{document}."
%
%===========================================================
\documentclass[11pt,twoside,titlepage]{article}
%%NEED TO ADD epsf!!
\usepackage{threeparttop}
\usepackage{graphicx}
\usepackage{latexsym}
\usepackage{color}
\usepackage{listings}
\usepackage{fancyvrb}
%\usepackage{pgf,pgfarrows,pgfnodes,pgfautomata,pgfheaps,pgfshade}
\usepackage{tikz}
\usepackage[normalem]{ulem}
\tikzset{
    %Define standard arrow tip
%    >=stealth',
    %Define style for boxes
    oval/.style={
           rectangle,
           rounded corners,
           draw=black, very thick,
           text width=6.5em,
           minimum height=2em,
           text centered},
    % Define arrow style
    arr/.style={
           ->,
           thick,
           shorten <=2pt,
           shorten >=2pt,}
}
\usepackage[noend]{algorithmic}
\usepackage[noend]{algorithm}
\newcommand{\bfor}{{\bf for\ }}
\newcommand{\bthen}{{\bf then\ }}
\newcommand{\bwhile}{{\bf while\ }}
\newcommand{\btrue}{{\bf true\ }}
\newcommand{\bfalse}{{\bf false\ }}
\newcommand{\bto}{{\bf to\ }}
\newcommand{\bdo}{{\bf do\ }}
\newcommand{\bif}{{\bf if\ }}
\newcommand{\belse}{{\bf else\ }}
\newcommand{\band}{{\bf and\ }}
\newcommand{\breturn}{{\bf return\ }}
\newcommand{\mod}{{\rm mod}}
\renewcommand{\algorithmiccomment}[1]{$\rhd$ #1}
\newenvironment{checklist}{\par\noindent\hspace{-.25in}{\bf Checklist:}\renewcommand{\labelitemi}{$\Box$}%
\begin{itemize}}{\end{itemize}}
\pagestyle{threepartheadings}
\usepackage{url}
\usepackage{wrapfig}
% removing the standard hyperref to avoid the horrible boxes
%\usepackage{hyperref}
\usepackage[hidelinks]{hyperref}
% added in the dtklogos for the bibtex formatting
\usepackage{dtklogos}
%=========================
% One-inch margins everywhere
%=========================
\setlength{\topmargin}{0in}
\setlength{\textheight}{8.5in}
\setlength{\oddsidemargin}{0in}
\setlength{\evensidemargin}{0in}
\setlength{\textwidth}{6.5in}
%===============================
%===============================
% Macro for document title:
%===============================
\newcommand{\MYTITLE}[1]%
   {\begin{center}
     \begin{center}
     \bf
     CMPSC 280\\Principles of Software Development\\
     Fall 2015
     \medskip
     \end{center}
     \bf
     #1
     \end{center}
}
%================================
% Macro for headings:
%================================
\newcommand{\MYHEADERS}[2]%
   {\lhead{#1}
    \rhead{#2}
    %\immediate\write16{}
    %\immediate\write16{DATE OF HANDOUT?}
    %\read16 to \dateofhandout
    \def \dateofhandout {September 18, 2015}
    \lfoot{\sc Handed out on \dateofhandout}
    %\immediate\write16{}
    %\immediate\write16{HANDOUT NUMBER?}
    %\read16 to\handoutnum
    \def \handoutnum {5}
    \rfoot{Handout \handoutnum}
   }

%================================
% Macro for bold italic:
%================================
\newcommand{\bit}[1]{{\textit{\textbf{#1}}}}

%=========================
% Non-zero paragraph skips.
%=========================
\setlength{\parskip}{1ex}

%=========================
% Create various environments:
%=========================
\newcommand{\PURPOSE}{\par\noindent\hspace{-.25in}{\bf Purpose:\ }}
\newcommand{\SUMMARY}{\par\noindent\hspace{-.25in}{\bf Summary:\ }}
\newcommand{\DETAILS}{\par\noindent\hspace{-.25in}{\bf Details:\ }}
\newcommand{\HANDIN}{\par\noindent\hspace{-.25in}{\bf Hand in:\ }}
\newcommand{\SUBHEAD}[1]{\bigskip\par\noindent\hspace{-.1in}{\sc #1}\\}
%\newenvironment{CHECKLIST}{\begin{itemize}}{\end{itemize}}


\usepackage[compact]{titlesec}

\begin{document}
\MYTITLE{Laboratory Assignment Six: Specifying and Implementing a Next Release Planner}
\MYHEADERS{Laboratory Assignment Six}{Due: October 23, 2015}

\section*{Introduction}

In the previous laboratory assignments, you have learned about the strategies that we will use throughout this semester
to specify, design, implement, test, and document Java programs.  You have also had several experiences with working in
progressively larger teams to complete the phases of the software life cycle.  Moreover, our past class sessions have
introduced you to the key concepts associated with software engineering, with a recent focus on the elicitation of
software requirements.  In this assignment, you and your team will follow the phases of the software development life
cycle and employ the concepts that we have studied in class to implement and test a program that can efficiently
determine how to release the next version of a software product. At the assignment's due date, you and your team will
describe and demonstrate your planner in a short presentation.

\section*{Next Release Planning}

Managers often use a software process in order to make decisions about when to release an application.  If you were a
manager and you were charged with determining which features to include in the next release of a program you could
use a next release planner to efficiently make an intelligent decision.  For this laboratory assignment, you should
assume that a next release planner accepts as input for each potential requirement $R_j$, (i) $C_j$, the cost associated with
implementing the requirement and (ii) $B_j$, the monetary benefit for a program that contains this feature.

In order to determine which requirements will be part of the next release of a software application, a manager must
choose from the requirements $R = \{ R_1, \ldots, R_n \}$ and ensure that (i) the implementation tasks are completed at
no more than the total fixed cost $C$ and (ii) the chosen requirements maximize the total monetary benefit that the
company will see when it releases the software product.  Given cost and benefit information for each $R_j$ and the fixed
cost $C$, a next release planner will pick which requirements are included in the next release of the program.

Next release planning is ``equivalent'' to a well-known NP-hard combinatorial optimization problem.  What is the
``equivalent'' problem? What are the heuristics for solving this problem?

\section*{Observations About the Requirements}

You are responsible for implementing a next release planner that should adhere to the requirements outlined in the
previous section.  Your planner should be implemented in Java and execute as a stand-alone program on the command-line.
It must accept command-line arguments for (i) the set of requirements $R = \{ R_1, \ldots, R_n \}$, (ii) the set of
benefits $B = \{ B_1, \ldots, B_n \}$, (iii) the set of costs $C = \{ C_1, \ldots, C_n \}$, and (iv) the
fixed cost $C$. Unless you decide otherwise, the planner may assume that all of the values in $R$, $B$, and $C$ are
integers. All of the command-line arguments must be recognized, verified, and parsed using the JCommander tool available
at \url{http://jcommander.org/}. The system must be able to efficiently solve large instances of the next release problem.

You must specify, as formally as is possible, the next release planner that your team will implement.  Whenever it is
justifiable to do so, your requirements document must adhere to the IEEE standard for software specification and answer
the questions in Table 4.2 of SETP.  Through team discussions and interactions with the course instructor, you must
define correct requirements that adhere to all of the characteristics of good requirements.  For instance, beyond being
correct and consistent, your requirements also must be testable and traceable.  Using Markdown --- or \LaTeX\ if that is
a better tool for clearly stating your requirements --- you should write a requirements document that fully explains the
inputs, outputs, and behavior of the next release planner.

\section*{Designing the System}

Working with the members of your team and leveraging the content in the requirements document, you must create a design
for your system.  As you are finalizing the object-oriented design, you should try to develop answers to relevant
questions such as: How many classes will you use? What will be the relationship between the classes? What methods will
the classes have? What will be the inputs and outputs of the methods?  Is the design testable?  After answering these
questions, you should use Markdown or \LaTeX\ to write a design document with text and diagrams that explain the system.
Adhering to the JavaDoc standard, you should create documentation for your classes.

\section*{Implementing and Testing the Program}

Using the requirements and design document, your team must implement and test the next release planner. You should focus on
implementing a next release planner that is both correct and efficient. Just like in the previous laboratory assignments,
your implementation must include the following:

\vspace*{-.1in}
\begin{enumerate}
\itemsep0em
    \item A version control repository that contains all of the artifacts for the project
    \item A high-quality JUnit test suite that effectively tests all of the classes in the program
    \item Fully documented Java source code that completely fulfills the stated requirements
\end{enumerate}
\vspace*{-.1in}

Since you cannot exhaustively test this application, you must decide what types of inputs you will create in the test
cases.  You will also need to determine how you will know that the output of the next release planner is correct.
Finally, make sure that your tests cover all of the requirements.

\section*{Summary of the Required Deliverables}

This assignment invites your team to submit one printed version of the following files:
\vspace*{-.1in}
\begin{enumerate}
    \itemsep0em
    \item A description of and justification for your team's chosen organization, roles, and tool support
    \item A document that clearly specifies the inputs, outputs, and behavior of the next release planner
    \item A document that explains the planner's design, with details about classes and methods
    \item All of the implementation artifacts (e.g., source code and its comprehensive documentation)
    \item A short five to ten minute presentation explaining and demonstrating your final system
\end{enumerate}
\vspace*{-.1in}

You must also ensure that the instructor has read access to your Bitbucket repository that is named according to the
convention {\tt cs280F2015-lab06-team{\em k}}, with {\tt {\em k}} representing the number of your assigned team.  Your
repository should contain all of the deliverables that you produced during the completion of this assignment.  Please
see the course instructor if you have any questions.

\end{document}
