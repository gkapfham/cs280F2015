% Typical usage (all UPPERCASE items are optional):
%       \input 580pre
%       \begin{document}
%       \MYTITLE{Title of document, e.g., Lab 1\\Due ...}
%       \MYHEADERS{short title}{other running head, e.g., due date}
%       \PURPOSE{Description of purpose}
%       \SUMMARY{Very short overview of assignment}
%       \DETAILS{Detailed description}
%         \SUBHEAD{if needed} ...
%         \SUBHEAD{if needed} ...
%          ...
%       \HANDIN{What to hand in and how}
%       \begin{checklist}
%       \item ...
%       \end{checklist}
% There is no need to include a "\documentstyle."
% However, there should be an "\end{document}."
%
%===========================================================
\documentclass[11pt,twoside,titlepage]{article}
%%NEED TO ADD epsf!!
\usepackage{threeparttop}
\usepackage{graphicx}
\usepackage{latexsym}
\usepackage{color}
\usepackage{listings}
\usepackage{fancyvrb}
%\usepackage{pgf,pgfarrows,pgfnodes,pgfautomata,pgfheaps,pgfshade}
\usepackage{tikz}
\usepackage[normalem]{ulem}
\tikzset{
    %Define standard arrow tip
%    >=stealth',
    %Define style for boxes
    oval/.style={
           rectangle,
           rounded corners,
           draw=black, very thick,
           text width=6.5em,
           minimum height=2em,
           text centered},
    % Define arrow style
    arr/.style={
           ->,
           thick,
           shorten <=2pt,
           shorten >=2pt,}
}
\usepackage[noend]{algorithmic}
\usepackage[noend]{algorithm}
\newcommand{\bfor}{{\bf for\ }}
\newcommand{\bthen}{{\bf then\ }}
\newcommand{\bwhile}{{\bf while\ }}
\newcommand{\btrue}{{\bf true\ }}
\newcommand{\bfalse}{{\bf false\ }}
\newcommand{\bto}{{\bf to\ }}
\newcommand{\bdo}{{\bf do\ }}
\newcommand{\bif}{{\bf if\ }}
\newcommand{\belse}{{\bf else\ }}
\newcommand{\band}{{\bf and\ }}
\newcommand{\breturn}{{\bf return\ }}
\newcommand{\mod}{{\rm mod}}
\renewcommand{\algorithmiccomment}[1]{$\rhd$ #1}
\newenvironment{checklist}{\par\noindent\hspace{-.25in}{\bf Checklist:}\renewcommand{\labelitemi}{$\Box$}%
\begin{itemize}}{\end{itemize}}
\pagestyle{threepartheadings}
\usepackage{url}
\usepackage{wrapfig}
% removing the standard hyperref to avoid the horrible boxes
%\usepackage{hyperref}
\usepackage[hidelinks]{hyperref}
% added in the dtklogos for the bibtex formatting
\usepackage{dtklogos}
%=========================
% One-inch margins everywhere
%=========================
\setlength{\topmargin}{0in}
\setlength{\textheight}{8.5in}
\setlength{\oddsidemargin}{0in}
\setlength{\evensidemargin}{0in}
\setlength{\textwidth}{6.5in}
%===============================
%===============================
% Macro for document title:
%===============================
\newcommand{\MYTITLE}[1]%
   {\begin{center}
     \begin{center}
     \bf
     CMPSC 280\\Principles of Software Development\\
     Fall 2015
     \medskip
     \end{center}
     \bf
     #1
     \end{center}
}
%================================
% Macro for headings:
%================================
\newcommand{\MYHEADERS}[2]%
   {\lhead{#1}
    \rhead{#2}
    %\immediate\write16{}
    %\immediate\write16{DATE OF HANDOUT?}
    %\read16 to \dateofhandout
    \def \dateofhandout {September 18, 2015}
    \lfoot{\sc Handed out on \dateofhandout}
    %\immediate\write16{}
    %\immediate\write16{HANDOUT NUMBER?}
    %\read16 to\handoutnum
    \def \handoutnum {5}
    \rfoot{Handout \handoutnum}
   }

%================================
% Macro for bold italic:
%================================
\newcommand{\bit}[1]{{\textit{\textbf{#1}}}}

%=========================
% Non-zero paragraph skips.
%=========================
\setlength{\parskip}{1ex}

%=========================
% Create various environments:
%=========================
\newcommand{\PURPOSE}{\par\noindent\hspace{-.25in}{\bf Purpose:\ }}
\newcommand{\SUMMARY}{\par\noindent\hspace{-.25in}{\bf Summary:\ }}
\newcommand{\DETAILS}{\par\noindent\hspace{-.25in}{\bf Details:\ }}
\newcommand{\HANDIN}{\par\noindent\hspace{-.25in}{\bf Hand in:\ }}
\newcommand{\SUBHEAD}[1]{\bigskip\par\noindent\hspace{-.1in}{\sc #1}\\}
%\newenvironment{CHECKLIST}{\begin{itemize}}{\end{itemize}}


\usepackage[compact]{titlesec}

\begin{document}
\MYTITLE{Laboratory Assignment Four: Team-Based Creation and Presentation of Software}
\MYHEADERS{Laboratory Assignment Four}{Due: September 25, 2015}

\vspace*{-.1in}
\section*{Introduction}

Now that you know how to use a wide variety of software tools and you are more comfortable with eliciting requirements
from a customer and then specifying, designing, implementing, validating, verifying, documenting, and releasing a
programming systems product, we will continue these efforts for a real-world system that is clearly useful to other
computer scientists and software engineers.

As you complete this assignment, please continue to pay careful attention to both the phases of the software development
lifecycle and the roles of the members of a software development team, as explained in Sections 1.5 through 1.8 of the
textbook.  Additionally, all of your team members should carefully review the content at the end of Section 2.2 in SETP
to learn more about the tasks associated with the validation and verification of software.  Next, your team members
should study the content in Section 3.1 so that they can learn more about project planning and the best way to define
project milestones and activities. Using the knowledge from this content in SETP, your team members should all be able
to outline the activities and milestones that are relevant for the completion of this assignment in one week. Finally,
they can examine the content in Section 3.2 of SETP to learn more about the roles in a software engineering project and the
working styles that team members may exhibit during the completion of this assignment.

It is also important for your team members to study Chapter 1 of MMM so that you can learn more about the effort that is
necessary to implement a programming systems product --- the ultimate goal that you should complete for this
assignment. Finally, please examine the content in MMM's Chapter 2 so that you understand the different types of tasks
that are normally completed during software development and how knowledge of a task's characteristics may better enable
you to estimate completion times. Please see the instructor if you have questions about these readings.

\section*{Organizing Your Software Development Team}

Please organize yourselves into teams of precisely three students. Whenever it is possible to do so, please make sure that you are
working with individuals who were not members of your team in the second laboratory assignment. After reviewing the
reading assignments mention in the previous section and discussing the software that you are invited to implement for
this assignment, your team should discuss who will complete this project's tasks. As you are making task assignments,
please think about the strengths and weaknesses of the members of your development team. For instance, you should ensure
that, as best as is possible, you ask the ``rational introverts'' on your team to complete tasks that are most suited to
their personalities. When it seems as though there are no team members who best fit certain roles, you should make
compromises to ensure that all work will still be successfully finished. Finally, your team should ensure that all of
your members know how to effectively use Git and Slack and then make a plan for how you will control your source code
and documentation and communication using channel messages and integrations.

\section*{Eliciting Requirements from the Customer}

For this laboratory assignment, your team is tasked with specifying, designing, implementing, documenting, and releasing
a programming systems product that will give users insights into all of the Git projects in their filesystem. For the
same of simplicity, this assignment sheet will refer to this product as {\tt git-beagle} because it will ``retrieve''
the important details about all of the Git repositories that are stored in a user's directories; each team is encouraged
to develop their own name for their programming systems product that best represents their tool's key features.

You should interact with the course instructor to learn more about the features that {\tt git-beagle} should provide.
Additionally, one of your team members should investigate existing tools or online discussion forums that outline the
features that are common to software tools like {\tt git-beagle}. However, the basic function of the tool will proceed
in the following manner. When given a root file system directory, {\tt git-beagle} will recursively traverse the
filesystem and look for all of the directories that contain local or remote Git repositories. After finding a Git
repository, {\tt git-beagle} will collect and report relevant information about this repository. For instance, your tool
might report files that have been modified but not yet committed and files that are currently in the directory but
not tracked by Git. If {\tt git-beagle} finds a Git repository that is ``clean'' and thus has nothing to commit, it
should also report this information to the user. The {\tt git-beagle} program should also report summary statistics that
note details about, for instance, how many repositories were found overall and how many files were in the repositories
but not yet added to Git's records.

For this laboratory assignment, all of the students responsible for requirements elicitation and analysis will interact
with the customer at the same time. In addition, these students may assume, for this assignment, that the customer is a
computer scientist who understands many of the intricacies associated with using the Git version control. When a team
learns something important about the features of the system, then one of this team's members should share these details
with all of the other team members through an appropriate channel in Slack. Since the ultimate goal for our course is to
release a final version of {\tt git-beagle} in a subsequent assignment, teams can share details about the system's
requirements within the appropriate bounds of the Honor Code.

\section*{Designing the System}

For this assignment, teams are encouraged to design and implement a system that uses whatever programming languages,
development environments, and execution means that are most suited to the knowledge and skills of the team's members. 
In light of this freedom, the language used in this section and the following sections will be ``generic'' and thus not,
for instance, refer to language concepts (e.g., ``object-oriented'' or ``Java class'') that are specific to a single
programming language or development environment. Please see the course instructor if you have questions about this
issue.

Working with the members of your team and leveraging the content in the requirements document, you should create a
design for your system.  As you are finalizing the system's design, you should try to develop answers to relevant
questions such as: How many components will you use? What will be the relationship between the components? What
functions will the components have? What will be the inputs and outputs of the functions?  Is the design easy to
understand?  Can you actually implement and test this design? After answering these questions, you should use Markdown
to write a design document with text and diagrams that explain the system. Since the main focus of this assignment is
not design diagrams, students may use any reasonable tool to create their diagrams and, if necessary, include them as
separate files in their team's repository. When clarification of the system's specification is necessary (e.g., due to
writing ambiguities), the designers should interact with the individual(s) who elicited \mbox{the requirements}. Yet,
your team should carefully ensure that the system's requirements and design are correctly in sync.

\section*{Implementing and Testing the System}

The developer(s) and tester(s) on your team should take the requirements and design documents and start to think about
how the system will be implemented. Your task is to ensure that the program faithfully adheres to the descriptions
already produced by other members of your team. When an aspect of the requirements and design documents is not clear,
the developer(s) must talk with the people who created these documents to resolve any outstanding concerns. The creators
of these documents must quickly commit any changes in them to their repository so as to best ensure that the
requirements and design of the system are in sync with its implementation.

As this programming systems product will be released to GitHub in a subsequent assignment, the implementors and testers
should take care to create a system that is well-documented through comments and other relevant documentation. Whenever
it is possible to do so, these members of your team must also add (semi-)automated methods for verifying that the
implementation adheres to the requirements that previously were elicited from the customer. For instance, if your team
decides to implement {\tt git-beagle} in the Java programming language, then you should test it with automated tests
cases that you wrote in JUnit, the industry standard for automated testing in Java. Overall, as you are implementing and
testing your system you should hold yourselves to a high standard under the assumption that other software engineers
will review and use your code.

\section*{Ensuring the Effective Operation of Your Team}

When you start to work on this laboratory assignment, it may seem as though the designer(s), developer(s), and tester(s)
``do not have any work to do'' because the requirements of the system have not been established. Yet, if you carefully
think about the work that you must complete for this assignment, it will become clear that this is not the case! For
instance, one member of your team should be tasked with creating all of the needed means for communication with tools
such as Slack and Bitbucket. Additionally, team members who are waiting to complete their chosen tasks should consider
investigating the use of tools, like Trello, to organize their team's efforts. Finally, it is important for team
members to spend time creating the templates for their deliverables and then carefully ``sharpening their tools''. For
instance, the developer(s) on the team can ensure that they have a smoothly functioning development environment that
will support the implementation of well-documented and correct code. Please see the instructor if you have questions
about \mbox{this matter}.

\section*{Presenting Your Programming Systems Product}

At the start of next week's laboratory session, each team will give a short five minute presentation explaining their
own implementation of the {\tt git-beagle} concept. You will be responsible for highlighting the key features of your
tool and the ways in which you did your best to specify, design, implement, release, and document it. Whenever possible,
you should give a high-quality, interactive, and engaging presentation that is both fun, interesting, and technically
correct. Unless a team can demonstrate that it is not possible for them to implement their presentation as a program,
all teams should use frameworks like {\tt reveal.js}, {\tt big}, or {\tt beamer} to create their presentations. Creating
your presentations using one of these tools is ideal because it will allow your team to store all of its content in a
Git version control repository and thus ultimately be released with the rest of your programming systems product.
Students who would like to learn more about implementing presentations with one of these frameworks can study some of
the examples that are available in the course instructor's GitHub repositories available at {\tt
https://github.com/gkapfham/}.

\section*{Carefully Review the Honor Code}

The Academic Honor Program that governs the entire academic program at Allegheny College is described in the Allegheny
Academic Bulletin.  The Honor Program applies to all work that is submitted for academic credit or to meet non-credit
requirements for graduation at Allegheny College.  This includes all work assigned for this class (e.g., examinations,
  laboratory assignments, and the final project).  All students who have enrolled in the College will work under the Honor
Program.  Each student who has matriculated at the College has acknowledged the following pledge:

\vspace*{-.05in}
\begin{quote}
  I hereby recognize and pledge to fulfill my responsibilities, as defined in the Honor Code, and to maintain the
  integrity of both myself and the College community as a whole.
\end{quote}
\vspace*{-.05in}

\noindent It is understood that an important part of the learning process in any course, and particularly one in
computer science, derives from thoughtful discussions with teachers and fellow students.  Such dialogue is encouraged.
However, it is necessary to distinguish carefully between the student who discusses the principles underlying a problem
with others and the student who produces assignments that are identical to, or merely variations on, someone else's
work.  While it is acceptable for students in this class to discuss their deliverables with their classmates, items that
are nearly identical to the work of others will be taken as evidence of violating the Honor Code. In particular, make
sure that your team members do not inappropriately share deliverables with the members of \mbox{other teams}.

\section*{Summary of the Required Deliverables}

This assignment invites you to submit printed and signed versions of the following deliverables; please see the
instructor if you have questions about any of these items. Please make sure that your team creates a version control
repository in Bitbucket to store all of the deliverables for this assignment; since you will again select your own
teams, please create a repository with a numerically-based name and share it with the course instructor. Also note that
all of the written documents must be prepared with Markdown and, for submission, converted to PDF using {\tt pandoc}.

\vspace*{-.1in}
\begin{enumerate}
  \setlength{\itemsep}{0in}
  \item A complete description of the roles that each team member fulfilled.
  \item A requirements document that fully describes the features of your system.
  \item Expressed in writing and as a simple technical diagram, the design of your system.
  \item Well-documented source code that fulfills all of the system's requirements.
  \item A tutorial that explains how to use all of the features of your finished system.
  \item Individually completed and submitted reviews of all of the team members.
  \item An informative and interesting five-minute presentation that highlights your software.
\end{enumerate}
\vspace*{-.1in}

\end{document}
